% !TeX encoding = UTF-8 Unicode
% !TeX program = xelatex+makeindex+bibtex
% !TeX spellcheck = sk-SK
%
\documentclass[11pt,a4paper%,twoside
]{article}
\usepackage{fontspec}
\usepackage{amssymb}
\setmainfont[Numbers=Lining]{Linux Libertine O}
\newcommand{\noteDynamics}[2][1.3]{\raisebox{0.05ex}{\fontspec[Scale=#1]{Emmentaler-11}{#2}}}
\usepackage{polyglossia}
\setdefaultlanguage{slovak}
\setotherlanguage{english}
\setotherlanguage{french}
%\usepackage{tocloft}
%\setlength\cftparskip{-2pt}
%\setlength\cftbeforesecskip{0.5em}
%\setlength\cftaftertoctitleskip{2pt}
\usepackage{xcolor}
\usepackage{natbib}
\usepackage{hyperref}
\definecolor{printblue}{rgb}{0,0,0.8}
\definecolor{proofreadcolor}{rgb}{0.0157,0.5176,0.2666}
\definecolor{attentioncolor}{rgb}{0.8235,0.2666,0}
\hypersetup{bookmarksnumbered=true,
bookmarksopen=false,
pdfpagemode=UseNone,
hypertexnames=false,
colorlinks=true,
urlcolor=printblue,
filecolor=printblue,
linkcolor=printblue,
citecolor=printblue,
anchorcolor=printblue,
hypertexnames=false,  
pdftitle={Dodatok ku knihe Základy cvičenia na klavíri},
pdfauthor={Chuan C. Chang},
pdfsubject={Metódy a techniky cvičenia na klavíri},
pdfkeywords={klavír, piano, cvičenie, technika, ladenie klavíra},
unicode=true,
pdfstartview=FitH}
\usepackage[left=1in,top=1in,right=1in,bottom=1in%,bindingoffset=0.5in
]{geometry}
\makeatletter
\DeclareRobustCommand{\em}{\@nomath\em\if\expandafter\@car\f@series\@nil\normalfont\else\it\bfseries\fi}
\makeatother
\title{Dodatok ku knihe \emph{Základy cvičenia na klavíri 3. edícia}}
\author{Chuan C. Chang}
\date{pôvodný dokument z 29. decembra 2019,\\preklad z \today}
\begin{document}
\maketitle
\begin{center}Translation \copyright\ \the\year\ Ladislav Rado, \href{mailto:lr@rsw.sk}{lr@rsw.sk}
\end{center}

\tableofcontents

\section{Recenzie kníh}
Cienniwa, Paul, By Heart: THE ART OF MEMORIZING MUSIC, 2014, 93 strán, register, no references (bibliography).
Je to trochu užitočná kniha; Pre knihu napísanú v roku 2014 však neexistujú žiadne odkazy, čo naznačuje nedostatok dostatočného výskumu, pretože už existuje množstvo publikácií o memorovaní, ktoré sú podrobnejšie preskúmané. Výsledkom je, že sa materiál často blíži konečnému produktu, ale nie je tam celkom, chýbajú mu niektoré hlavné súčasti a obsahuje veľa materiálu, ktorý nesúvisí s ukladaním do pamäte. Napísať knihu výlučne o memorovaní je však možné, pretože memorovanie má vplyv na mnohé aspekty učenia sa a hry na klavíri.

Jeho metóda spočíva v rozdelení diela, ktoré sa má uložiť do pamäte, na krátke časti označené „orientačnými bodmi“. Tieto sekcie sa ukladajú do pamäte pri pomalej hre s metronómom, a to hraním aj pomocou mentálnej hry (mimo klavíra). Na boj proti nervozite sa odporúča meditácia. Proces memorovania rozdeľuje do troch etáp, ale v každom štádiu skutočne nejde o nič nové; oni sú
v zásade rovnaké postupy vykonávané na vyšších úrovniach dokonalosti.

Dobrá rada:
(1) spoliehať sa na hmatovú pamäť je zlý nápad, môže spôsobiť problémy; dostať sa preč od hmatu
pamäť, používajte pomalé cvičenie,
(2) Najprv si zapamätajte, potom sa to naučte (precvičujte),
(3) mentálna hra je zásadná pre zapamätanie si a predvádzanie sa a je neoddeliteľnou súčasťou hry
proces zapamätávania,
(4) nadmerné cvičenie vedie k mnohým problémom,
(5) čestnosť a morálka sú veľmi dôležité pre štúdium klavíra,
(6) neučte sa nové kúsky tesne pred predstavením,
(7) opakované predstavenia sú veľmi ťažké, takže si nehrajte na srdce tesne pred vystúpením,
(8) Nepokladajte hudbu k notám, pretože ich nebudete počuť
hranie a pod.
Všetky tieto rady sú v mojej knihe.

Zlá rada:
(1) počúvanie predstavení a záznamov je zlé pre učenie a pamäť,
(2) odporúča prílišné používanie metronómu
(3) nič o rutinách prípravy na výkon atď.
O týchto témach pojednáva aj moja kniha; existuje mnoho ďalších chýb, príkladov zlej logiky a nezrovnalostí

Toto nie je učebnica memorovania, ale správa jedného klaviristu o jeho skúsenostiach začínajúcich ako čembalo (žiadne memorovanie), prechod na klaviristu, ktorý musí hrať naspamäť. Aby ste z tejto knihy získali maximum, musíte byť schopní rozlíšiť dobré rady od nesprávnych.

Feldenkrais, Moshe, The Elusice Evivid, The Comnvergence of Movement, Neuroplasticity \& Health, 2019, 166 strán, bibliography, index. (v procese).

Alexander, F. M., The Use of Self, the Alexander Technique by its originator. (v procese). 

\section{Strata sluchu a ušný maz, Vertigo (návrh)}
(1) Ušný vosk môže spôsobiť infekcie a stratu sluchu; preto by mal každý navštíviť lekára ORL (ucho, nos, hrdlo), aby mu ucho vyšetril a urobil si sluchový test, a naučil by sa, ako si sám odstrániť ušný vosk. Lekári sa vždy mračia, keď používajú tipy q alebo nástroje na odstraňovanie ušného mazu, pretože časť ušného vosku, ktorá ovplyvňuje sluch, je spravidla vo vnútri zvukovodu, v blízkosti ušného bubienka. Akákoľvek snaha uvoľniť tento ušný vosk ho môže posunúť ďalej a zhoršiť situáciu.

Najpopulárnejším bezpečným spôsobom čistenia zvukovodu je použitie 3\% roztoku peroxidu vodíka. Ľahnite si vodorovne tak, aby jedno ucho smerovalo nahor. Vezmite si kvapkadlo do očí a do ucha vložte asi 0,5 cm3 peroxidu. Bežne môžete počuť bublanie, pretože peroxid reaguje s ušným voskom a vytvára bubliny:

\begin{equation}
m \times H_2O_2 + n \times (CH) \leadsto y \times CO_2 + z \times H_2O
\end{equation}

kde (CH) predstavuje ušný vosk. Ide o neškodnú reakciu, ktorá neškodí, ale peroxid bude s voskom reagovať a uvoľní ho. Asi po 5 minútach priložte k uchu hodvábny papier a
postavte sa a nechajte roztok vyliať a zachyťte ho tkanivom. Utrite čo najviac tekutiny a poznamenajte si, či z nej vyšiel vosk. Ucho môžete tiež opláchnuť teplou vodou (bezprostredne po
peroxidová liečba) pomocou žiarovky na výplachy uší, ktorú je možné získať v lekárni. Zatvorte odtok drezu tak, aby sa všetka voda z oplachovania zachytila ​​v dreze, aby sa v prípade, že sa uvoľnia veľké častice vosku
týmto spôsobom to budete môcť vidieť v umývadle. V poslednom kroku vysušte zvukovod špičkou q naklonením ucha nadol a zasunutím špičky q asi štvrť cm (0,5 cm) do ucha.

Tí, ktorí majú problémy s ušným voskom, zaznamenajú okamžité zlepšenie sluchu. Zvlášť, ak opláchnete vodou, môže sa vám po čistení trochu točiť hlava alebo sa vám bude točiť hlava, dávajte si preto pozor na rovnováhu, kým si nebudete istí, že vaša rovnováha nebola ovplyvnená. U osôb s tinnitom môže zlepšený sluch znížiť tinnitus, pretože keď sú všetky zvuky hlasnejšie, mozog svoje zosilnenie automaticky vypne.

Aj keď je to najbežnejšie používaná metóda, nemusí byť najpraktickejšia, pretože nepoužíva prirodzený mechanizmus odstraňovania ušného vosku z ucha. Alternatívna metóda využívajúca minerálny olej môže fungovať lepšie, hlavne preto, že je jednoduchšia. Ľahnite si jedným uchom nahor a dajte tri kvapky minerálneho oleja, počkajte päť minút, vstaňte a zotrite všetok olej, ktorý sa vyleje z ucha; potom opakujte pre druhé ucho, trikrát týždenne. To zmäkčuje ušný vosk, aby fungovali prirodzené pohyby chĺpkov v uchu, ktoré vosk vymetajú. Tomuto procesu môžete veľmi pomôcť tým, že pomocou q-tipov odstránite ušný vosk, ktorý sa hromadí v blízkosti otvoru ucha, vždy, keď sa osprchujete, pretože počas sprchovania voda nevyhnutne vstupuje do ucha a ďalej zmäkčuje ušný vosk, čo uľahčuje vyčistiť.
Netlačte hrot q hlboko do ucha, pretože by to mohlo spôsobiť zasunutie veľkých kúskov vosku späť do ucha alebo dokonca poškodenie ušného bubienka; Jednoducho položte hrot q na otvor a otočte ho medzi prstami, aby ste nasali všetku tekutinu.

(2) Takmer každý človek zažije vertigo aspoň raz za život. Pretože to môže byť taký oslabujúci a traumatický zážitok, ktorý môže mať za následok zbytočné núdzové návštevy nemocnice,
atď., Je užitočné vedieť, čo je to vertigo, než sa to stane.

Vertigo je porucha rovnovážneho mechanizmu v uchu, ktorá spôsobuje, že sa zorné pole točí okolo, čo spôsobuje nevoľnosť, vracanie a stratu rovnováhy. Len veľmi zriedka je smrteľný. Jediným okamžitým riešením je zavrieť oči a ľahnúť si do polohy, ktorá má za následok minimálne príznaky. Môžete byť schopní piť iba malé množstvo tekutín bez zvracania. Je to zvyčajne spôsobené infekciou, ktorá vytláča otolity z ich normálnych polôh. Prechladnutie atď. Môže spôsobiť vertigo a môže trvať jeden deň až niekoľko týždňov. Akonáhle zažijete vážny prípad, vaše šance na opätovné získanie sa výrazne znížia, pretože počiatočná infekcia vás imunizuje proti tomuto konkrétnemu infekčnému agensu. Ak zažijete ďalší incident, je zvyčajne oveľa miernejší.

V závažných prípadoch môže lekár predpísať lieky na zmiernenie nepohodlia spôsobeného nevoľnosťou a na liečbu infekcií alebo zápalov. Existuje množstvo článkov o závratoch, ako napríklad:

http://www.webmd.com/brain/vertigo-symptoms-causes-treatment

preto hľadajte na internete, ak potrebujete viac informácií. Na niektoré typy vertigo existujú domáce opravné prostriedky:

http://www.diziness-and-balance.com/disorders/bppv/home/home-pc.html

znova vyhľadajte najnovšie informácie na internete.

\end{document}